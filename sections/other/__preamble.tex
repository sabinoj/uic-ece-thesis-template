
% This is the user manual for UICTHESI CLS, originally found
% at https://www.math.uic.edu/graduate/current/uicthesi, and
% modified by Pete Snyder <snyder@gmail.com> to match the
% current department requirements.
\usepackage{booktabs}
\usepackage{listings}
\usepackage{newlfont}
\usepackage{amsfonts}
\usepackage{amssymb}
\usepackage{euler}
%\usepackage[xindy={glsnumbers=false},nonumberlist,acronym,nopostdot,nogroupskip,nomain]{glossaries}
\usepackage[acronym]{glossaries}
\usepackage{xspace}
\usepackage[htt]{hyphenat}
\usepackage{float}
\usepackage{flushend}
\usepackage{footnote}
\usepackage{enumitem}
\usepackage{url}
%\usepackage{caption} %Just taking this out to stop the warnings, it appears it does not need it to build the document.
\usepackage{graphicx}

%This is to give more utility to the bibliography sections:
%\usepackage[backend=biber]{biblatex}
%\addbibresource{bibtex/references}

%Packages from the JMEMS paper:
\usepackage{amsmath}
\usepackage{algorithmic}
\usepackage{algorithm}
\usepackage{array}
%\usepackage[caption=false,font=normalsize,labelfont=sf,textfont=sf]{subfig}
\usepackage{textcomp}
\usepackage{stfloats}
%\usepackage{url}
\usepackage{verbatim}
%\usepackage{graphicx}
\usepackage{cite}
\usepackage{siunitx}
\usepackage{cuted}
%\usepackage{float}


%\setacronymstyle{long-short}

%for the acronyms please remember that after you generate one a .acn file will be made, use the following command to compile it.
%makeindex John_Sabino_Thesis.acn -s John_Sabino_Thesis.ist -o John_Sabino_Thesis.gls
%Then re-compile the main Latex file twice to have it properly inserted for the List of abbreviations.

%This is from ChatGPT to make standard figures have double spaced text.
%\captionsetup{font={stretch=2}}

\newglossarystyle{clong}{%
 \renewenvironment{theglossary}%
     {\begin{longtable}{p{.2\linewidth}p{.8\linewidth}}}%
     {\end{longtable}}%
  \renewcommand*{\glossaryheader}{}%
  \renewcommand*{\glsgroupheading}[1]{}%
  \renewcommand*{\glossaryentryfield}[5]{%
    \glstarget{##1}{##2} & ##3\glspostdescription\space ##5\\}%
  \renewcommand*{\glossarysubentryfield}[6]{%
     & \glstarget{##2}{\strut}##4\glspostdescription\space ##6\\}%
  %\renewcommand*{\glsgroupskip}{ & \\}%
}

\makesavenoteenv{tabular}
\makesavenoteenv{table}

\makeglossaries
% See https://www.ctan.org/pkg/glossaries for questions on this package.
% Refer to the acronyms you define here as \gls{nyc}.
%
% This make sure that this text:
%    The Ramones are from \gls{nyc}, thats right, \gls{nyc}.
% Gets output like this:
%    The Ramones are from New York City (NYC), thats right, NYC.
% And that a line in the acronyms section of your thesis has an entry like:
%    NYC         New York City
\newacronym{ad}{AD}{Aerodynamic Diameter}
\newacronym{ald}{ALD}{Atomic Layer Deposition}
\newacronym{anl}{ANL}{Argonne National Laboratory}
\newacronym{cad}{CAD}{Computer-Aided Design}
\newacronym{cdc}{CDC}{Center for Disease Control}
\newacronym{ce}{CE}{Collection Efficiency}
\newacronym{cfd}{CFD}{Computational Fluid Dynamics}
\newacronym{cmos}{CMOS}{Complementary Metal Oxide Semiconductor}
\newacronym{cnm}{CNM}{Center for Nanoscale Materials}
\newacronym{cwp}{CWP}{Coal Worker's Pneumoconiosis}
\newacronym{cr}{Cr}{Chromium}
\newacronym{dcb}{DCB}{Down Conversion Board}
\newacronym{di}{DI}{Deionized}
\newacronym{dill}{DiLL}{Dip-in-Laser Lithography}
\newacronym{dlp}{DLP}{Digital Light Processing}
\newacronym{drms}{DRMS}{Dual-Resonator Mass Sensor}
\newacronym{epa}{EPA}{Environmental Protetction Agency}
\newacronym{fbar}{FBAR}{Film Bulk Acoustic Resonator}
\newacronym{fem}{FEM}{Finite Element Modeling}
\newacronym{fp}{FP}{Field Programmable}
\newacronym{fpga}{FPGA}{Field Programmable Gate Arrays}
\newacronym{fsm}{FSM}{Finite State Machine}
\newacronym{gcsr}{GCSR}{Global Control-Selective Response}
\newacronym{gpio}{GPIO}{General Purpose Input Output}
\newacronym{hepa}{HEPA}{High-Efficiency Particulate Air}
\newacronym{hf}{HF}{Hydrofloric acid}
\newacronym{i2c}{$I^2C$}{Inter-Integrated Circuit}
\newacronym{ic}{IC}{Integrated Circuit}
\newacronym{ipa}{IPA}{Isopropanol Alcohol}
\newacronym{ir}{IR}{Infrared}
\newacronym{ito}{ITO}{Indium-Tin Oxide}
\newacronym{lcd}{LCD}{Liquid Crystal Display}
\newacronym{lpcvd}{LPCVD}{Low Pressure Chemical Vapor Deposition}
\newacronym{mmd}{MMD}{Mass Median Aerodynamic Diameter}
\newacronym{mems}{MEMS}{MicroElectroMechanical Systems}
\newacronym{mosfet}{MOSFET}{Metal Oxide Semiconductor Field Effect Transistor}
\newacronym{msha}{MSHA}{Mine Safety and Health Administration}
\newacronym{msl}{MSL}{Micromechatronic Systems Laboratory}
\newacronym{msla}{mSLA}{Masked Stereolithography Apparatus}
\newacronym{n2}{$N_{2}$}{Nitrogen}
\newacronym{nc}{NC}{Normally Closed}
\newacronym{nhg}{NHG}{Nested Hysteresis Gap}
\newacronym{niosh}{NIOSH}{National Institute of Occupational Safety and Health}
\newacronym{no}{NO}{Normally Open}
\newacronym{opc}{OPC}{Optical Particle Counter}
\newacronym{osha}{OSHA}{Occupational Safety and Health Administration}
\newacronym{pbw}{PBW}{Particle-Bound Water}
\newacronym{pcb}{PCB}{Printed Circuit Board}
\newacronym{pdm}{PDM}{Personal Dust Monitor}
\newacronym{pel}{PEL}{Permissible Exposure Limit}
\newacronym{pfsm}{PFSM}{Physical Finite State Machine}
\newacronym{plc}{PLC}{Programmable Logic Controllers}
\newacronym{pll}{PLL}{Phase-Lock Loop}
\newacronym{pm}{PM}{Particulate Matter}
%\newacronym{pm2-5}{$PM_{2.5}$}{Particulate Matter less than 2.5 $\si{\micro\metre}$}
%\newacronym{pm10}{$PM_{10}$}{Particulte Matter less than 10 $\si{\micro\metre}$}
\newacronym{psl}{PSL}{Polystyrene Latex}
\newacronym{pvd}{PVD}{Physical Vapor Deposition}
\newacronym{pzt}{PZT}{Lead Zirconate Titanate}
\newacronym{qcm}{QCM}{Quartz Crystal Microbalance}
\newacronym{rcs}{RCS}{Respirable Crystalline Silica}
\newacronym{rf}{RF}{radio frequency}
\newacronym{rh}{RH}{Relative Humidity}
\newacronym{satc}{SAT-C}{SATurated-Cluster}
\newacronym{saw}{SAW}{Surface Acoustic Wave}
\newacronym{sda}{SDA}{Scratch-Drive Actuator}
\newacronym{sem}{SEM}{Scanning Electron Microscope}
\newacronym{si3n4}{$Si_{3}$$N_{4}$}{Silicon Nitride}
\newacronym{sio2}{$SiO_{2}$}{Silicon Dioxide}
\newacronym{sipo}{SIPO}{Serial In Parallel Out}
\newacronym{soc}{SoC}{System on Chip}
\newacronym{string}{STRING}{STRIctly Non-nested hysteresis Gaps }
\newacronym{teom}{TEOM}{Tapered Element Oscillating Microbalance}
\newacronym{2pp}{2PP}{Two Photon Polymerization}
\newacronym{uart}{UART}{Universal Asynchronous Receive Transmit}
\newacronym{uic}{UIC}{University of Illinois Chicago}
\newacronym{uv}{UV}{Ultra Violet}
\newacronym{vf}{$V_{f}$}{Flow Velocity}
\newacronym{vts}{$V_{TS}$}{Terminal Settling Velocity}
\newacronym{vi}{VI}{Virtual Impactor}
\newacronym{weardm}{WEARDM}{Wearable Respirable Dust Monitor}
\newacronym{xrd}{XRD}{X-Ray Diffraction}

\def\new@fontshape#1#2#3#4#5{\expandafter
     \edef\csname#1/#2/#3\endcsname{\expandafter\noexpand
                                 \csname #4\endcsname}}
\new@fontshape{cmr}{bx}{sc}{
      <5>cmcsc8 at 5pt%
      <6>cmcsc8 at 6pt%
      <7>cmcsc8 at 7pt%
      <8>cmcsc8%
      <9>cmcsc9%
      <10>cmcsc10%
      <11>cmcsc10 at 10.95pt%
      <12>cmcsc10 at 12pt%
      <14>cmcsc10 at 14.4pt%
      <17>cmcsc10 at 17.28%
      <20>cmcsc10 at 20.736pt%
      <25>cmcsc10 at 24.8832pt%
      }{}
\mathversion{normal}
\newcommand{\ams}{{$\cal{A}\cal{M}\cal{S}$}}
\newcommand{\amslatex}{{$\cal{A}\cal{M}\cal{S}$-\LaTeX{}}}
\newcommand{\amstex}{{$\cal{A}\cal{M}\cal{S}$-\TeX{}}}
\newcommand{\BibTeX}{{\rm B\kern-.05em{\sc i\kern-.025em b}\kern-.08em
    T\kern-.1667em\lower.7ex\hbox{E}\kern-.125emX}}
\newcommand{\uicthesi}{{$\mathbb{UICTHESI}$}}

\newcommand\bs{\char '134 }   % A backslash character for \tt font
\newcommand{\lb}{\char '173 } % A left brace character for \tt font
\newcommand{\rb}{\char '175 } % A right brace character for \tt font

% one or two other commands
\def\newfont#1#2{\@ifdefinable #1{\font #1=#2\relax}}
\def\symbol#1{\char #1\relax}

%\makeglossaries
